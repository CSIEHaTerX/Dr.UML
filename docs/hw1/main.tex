\documentclass[12pt]{article}

% refer to configurations.tex for the LaTeX setup of this template
\usepackage[english]{babel}
\usepackage{CJKutf8}
\title{\textbf{\textsf{Dr. UML}}}

% time table
\usepackage{longtable}

% math
\usepackage{amsmath, amssymb}

% references
\usepackage[style=apa, backend=biber]{biblatex}
\addbibresource{bibliography.bib}

% SVG images
\usepackage{svg}
\usepackage{amsmath}
\usepackage[inkscapelatex=false]{svg}


% fonts
\usepackage{helvet}
\usepackage{sectsty}
\allsectionsfont{\sffamily} % for section titltes use sans-serif
% \renewcommand{\familydefault}{\sfdefault} % comment out for sans-serif font
% \usepackage{sansmath} % comment out for sans-serif math font
% \sansmath % comment out for sans-serif math font

% margins
\usepackage{geometry}
\geometry{
  a4paper,
  total={170mm,257mm},
  left=25mm,
  right=25mm,
  top=30mm,
  bottom=30mm,
}

% no indentation when a new paragraph starts
\setlength{\parindent}{0cm}

% links
\usepackage{hyperref} % better links
\usepackage{color}    % nicer link colors
\definecolor{pigment}{rgb}{0.2, 0.2, 0.6}
\hypersetup{
  colorlinks = true, % Color links instead of ugly boxes
  urlcolor   = pigment, % Color for external hyperlinks
  linkcolor  = black, % Color for internal links
  citecolor  = pigment % Color for citations
}

% headers
\usepackage{fancyhdr}
\pagestyle{fancy}
\lhead{Quiddity: A collaborative UML white board}
\chead{}
\rhead{}

% example boxes
\usepackage{tcolorbox}
\newtcolorbox{examplebox}{
  colback=white,
  colframe=gray!30,
  title=Example,
  sharp corners,
  boxrule=0.5pt,
  coltitle=black
}

% conditionals
\usepackage{ifthen}
\newboolean{showinstructions}
\newboolean{showexamples}
\newboolean{showexplanations}
\renewenvironment{examplebox}{%
  \ifthenelse{\boolean{showexamples}}%
    {\begin{tcolorbox}[colback=white, colframe=gray!30, title=Example, sharp corners, boxrule=0.5pt, coltitle=black]}%
    {\expandafter\comment}%
}{%
  \ifthenelse{\boolean{showexamples}}%
    {\end{tcolorbox}}%
    {\expandafter\endcomment}%
}


\usepackage{changepage}
% \usepackage{lipsum} % just for the example

\newenvironment{subs}
  {\adjustwidth{3em}{0pt}}
  {\endadjustwidth}


% Define a new environment for explanations
\newcommand{\explanation}[1]{%
  \ifthenelse{\boolean{showexplanations}}%
    {\textit{Explanation:} #1}%
    {\ignorespaces}%
}

% Define a new environment for instructions
\newcommand{\instructions}[1]{%
  \ifthenelse{\boolean{showinstructions}}%
    {#1}%
    {\ignorespaces}%
}

\makeatletter
\newcommand{\maketitlepage}{%
    \begin{titlepage}
        \maketitle
        \thispagestyle{empty}
        \vfill 
        \centering
        \author{CSIE IV 蕭耕宏 110590005\\
                CSIE IV 黃冠鈞 110590028\\
                CSIE IV 張庭瑋 110590035\\
                CSIE IV 吳宥駒 110590066\\
                Homework \#2
                }
        \vfill 
    \end{titlepage}
    \newpage
}
\makeatother

% Optional user settings
\setboolean{showinstructions}{false} % set to false to hide instructions
\setboolean{showexamples}{false} % set to false to hide examples
\setboolean{showexplanations}{false} % set to false to hide explanations

\begin{document}

\begin{CJK*}{UTF8}{bsmi} % for Chinese
\maketitlepage
\end{CJK*}

\section{Problem statement}


There are several tools available for creating UML diagrams on the internet but many of them come with paid subscriptions or limitations that make them less accessible. Moreover, they often resort to creating poorly formatted documents due to the lack of affordable, high-quality options. As the result, we propose Quiddity.\\

Quiddity is an innovative collaborative platform designed for software developers, system architects, and students who need to efficiently create and manage Unified Modeling Language (UML) diagrams.\\

The tool meets the pressing need for collaborative designing by allowing teams to work together simultaneously, regardless of their physical location. Also, it offers real-time updates and integrated communication features. Quiddity will be used primarily in design meetings, brainstorming sessions, and technical workshops where immediate visual feedback is essential. It is needed when precise and dynamic visual representation of complex systems is required to align team understanding and streamline development processes.\\

Quiddity integrates a robust set of customizable UML elements with drag-and-drop functionality and real-time collaboration. This not only enhances the creative aspects of system design but also ensures that technical requirements are met with precision and clarity. The platform's intuitive design and collaborative capabilities make it an essential tool for modern software development teams, system architects, and students aiming to create high-quality UML diagrams efficiently.

\section{The developemeent language}
Golang.

\section{Measurement}
    
\begin{itemize}
  \item{2025/02/20 17:00\textasciitilde18:20}
\end{itemize}


%\newpage
%\section*{References}
\nocite{Siepe2024}
\printbibliography[heading=none]

\end{document}